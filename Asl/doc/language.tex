\section{Asl}

\asl\ is a very simple imperative programming language created to illustrate
how an interpreter works. The \asl\ interpreter has been designed using
\href{http://www.antlr.org}{ANTLR v3} and Java.
Let us start with a simple example written in \asl.

\begin{verbatim}
    /**
     * A function that computes the factorial of a number
     * using a recursive definition of factorial.
     */
    func factorial(n)
      if n <= 1 then return 1 endif;
      return n*factorial(n-1)
    endfunc

    // The main program
    func main()
      write "Enter a number: "; read i;
      write "The factorial of "; write i;
      write " is: "; write factorial(i);
      write "%n"
    endfunc
\end{verbatim}


The program has two functions: \texttt{factorial} and \texttt{main}.
Every program must have a \texttt{main} function without parameters
that is identified as the primary function where the program starts
executing.

\subsection{Data types and variables}

\asl\ only has two data types: \emph{integer} and \emph{Boolean}.
It does not have any complex data structures such as arrays or structs
(that is why the language is so simple).

\asl\ is a typeless programming language, i.e., any variable can have
values of any type and the type can change dynamically. As in many
interpreted languages, variables are not declared. They are created
when a value is assigned to them. Type checking is performed dynamically
at runtime.

\asl\ does not have global variables. All the variables are local within
the function in which they are created. The only communication mechanism
between functions is parameter passing.

Functions are also typeless and they can return any type of value
(or nothing) depending on their execution.

\subsection{Assignment}

It is the basic statement of an imperative language. The syntax is
the following:

\begin{quote}
\texttt{variable = }\emph{expression}
\end{quote}

If the variable does not exist, it is created by the assignment.
No type checking is performed with regard to the previous value of
the variable since variables can change their type dynamically.

\subsection{Input and output}

\asl\ can only read from the standard input and write to the standard
output. The \texttt{read} statement can only read integer values into
a variable, whereas the \texttt{write} statement can either write
a string or the value of an expression. The strings are interpreted
in the same way as in Java using the \texttt{\%} conversion. For example,
\texttt{\%n} is converted into a newline.

\subsection{Control flow statements}

The following program covers all the existing statements in the language.
The function \texttt{is\_prime} returns a Boolean result indicating
whether \texttt{n} is prime or not. In case it is not prime, it also writes
the smallest prime divisor of the number.

\begin{verbatim}
    func main()
      write "Enter a number: "; read x;
      d = 1;
      p = is_prime(x,d);
      if p then write "It is prime.%n"
      else write "It is not prime.%n" endif;
      
      if not p then
        write d;
        write " is a divisor of ";
        write x; write ".%n"
      endif
    endfunc
    
    func is_prime(n, &div)
      if n = 1 then return false endif;
      div = 2;
      while div*div <= n do
        if n%div = 0 then return false endif;
        div = div + 1;
      endwhile;
      return true
    endfunc
\end{verbatim}


\noindent There are three statements for control flow:
\begin{quote}
\texttt{if} \emph{expression} \texttt{then} \emph{list\_instructions}
\texttt{else} \emph{list\_instructions} \texttt{endif}\\
\\
\texttt{while} \emph{expression} \texttt{do} \emph{list\_instructions}
\texttt{endwhile}\\
\\
\texttt{return} \emph{expression}
\end{quote}

\noindent The \texttt{else} clause of the \texttt{if} statement is optional.
The semantics of these statements is the same as the one of
similar instructions in other programming languages.

The semicolon \texttt{';'} is used to separate individual statements.
Note that the semicolon is used to separate statements and not to
terminate them. However, a semicolon after a final statement is also
accepted and interpreted as a separator from an empty instruction.

\subsection{Parameters}

Parameters are also typeless. There are two mechanisms to pass parameters:
by \emph{value} and by \emph{reference}. The mechanism is distinguished
by adding the prefix \texttt{\&} in the declaration of reference
parameters.

In the previous example, the funcion \texttt{is\_prime} has one
parameter passed by reference (\texttt{\&div}). The caller does not have
to specify the parameter passing is by value or reference. The
mechanism is implicitly inferred from the declaration of parameters
in the callee.

Expressions can be passed by value, whereas only variables can be
passed by reference.

\subsection{Expressions}

Expressions are written as sequences of operands and operators.
\asl\ has a rich set of logical, relational and arithmetic operators.
The following table contains all the operators in precedence order.
The topmost operators are the ones with highest priority.

\begin{center}
\begin{tabular}{ll}
logical negation and unary sign & \texttt{not}, \texttt{+}, \texttt{-} \\
multiplicative arithmetic operations & \texttt{*}, \texttt{/}, \texttt{\%}\\
additive arithmetic operations & \texttt{+}, \texttt{-}\\
relational operators & \texttt{=}, \texttt{!=}, \texttt{<}, \texttt{<=}, \texttt{>}, \texttt{>=}\\
logical and & \texttt{and}\\
logical or & \texttt{or}
\end{tabular}
\end{center}

Type checking is done when evaluating expressions. The arithmetic
operators require the operands to be integer, whereas the logical
operators require the operands to be Boolean. Relational operators
require both operands to have the same type. When comparing Boolean
values, \emph{true} is assumed to be greater than \emph{false}.

